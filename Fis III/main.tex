%%%%%%%%%%%%%%%%%%%%%%%%%%%%%%%%%%%%%%%%%
% kaobook
% LaTeX Template
% Version 1.2 (4/1/2020)
%
% This template originates from:
% https://www.LaTeXTemplates.com
%
% For the latest template development version and to make contributions:
% https://github.com/fmarotta/kaobook
%
% Authors:
% Federico Marotta (federicomarotta@mail.com)
% Based on the doctoral thesis of Ken Arroyo Ohori (https://3d.bk.tudelft.nl/ken/en)
% and on the Tufte-LaTeX class.
% Modified for LaTeX Templates by Vel (vel@latextemplates.com)
%
% License:
% CC0 1.0 Universal (see included MANIFEST.md file)
%
%%%%%%%%%%%%%%%%%%%%%%%%%%%%%%%%%%%%%%%%%

%----------------------------------------------------------------------------------------
%	PACKAGES AND OTHER DOCUMENT CONFIGURATIONS
%----------------------------------------------------------------------------------------

\documentclass[
	fontsize=10pt, % Base font size
	twoside=false, % Use different layouts for even and odd pages (in particular, if twoside=true, the margin column will be always on the outside)
	%open=any, % If twoside=true, uncomment this to force new chapters to start on any page, not only on right (odd) pages
	%chapterprefix=true, % Uncomment to use the word "Chapter" before chapter numbers everywhere they appear
	%chapterentrydots=true, % Uncomment to output dots from the chapter name to the page number in the table of contents
	numbers=noenddot, % Comment to output dots after chapter numbers; the most common values for this option are: enddot, noenddot and auto (see the KOMAScript documentation for an in-depth explanation)
	%draft=true, % If uncommented, rulers will be added in the header and footer
	%overfullrule=true, % If uncommented, overly long lines will be marked by a black box; useful for correcting spacing problems
]{kaobook}

% Set the language
\usepackage[english]{babel} % Load characters and hyphenation
\usepackage[english=british]{csquotes} % English quotes

% Load packages for testing
\usepackage{blindtext}
%\usepackage{showframe} % Uncomment to show boxes around the text area, margin, header and footer
%\usepackage{showlabels} % Uncomment to output the content of \label commands to the document where they are used

% Load the bibliography package
\usepackage{styles/kaobiblio}
\addbibresource{main.bib} % Bibliography file

% Load mathematical packages for theorems and related environments. NOTE: choose only one between 'mdftheorems' and 'plaintheorems'.
\usepackage{styles/mdftheorems}
%\usepackage{styles/plaintheorems}

\graphicspath{{examples/documentation/images/}{images/}} % Paths in which to look for images

\makeindex[columns=3, title=Alphabetical Index, intoc] % Make LaTeX produce the files required to compile the index

\makeglossaries % Make LaTeX produce the files required to compile the glossary

\makenomenclature % Make LaTeX produce the files required to compile the nomenclature

% Reset sidenote counter at chapters
%\counterwithin*{sidenote}{chapter}

%----------------------------------------------------------------------------------------

\begin{document}

%----------------------------------------------------------------------------------------
%	BOOK INFORMATION
%----------------------------------------------------------------------------------------

\titlehead{}
\subject{Series of Notes on Introductory Physics}

\title{An introduction to Electricity and Magnetism}
\subtitle{Class Notes ????}

\author[Sasa Salmen]{Sasa Salmen}

\date{\today}

\publishers{CCM-USP}

%----------------------------------------------------------------------------------------

\frontmatter % Denotes the start of the pre-document content, uses roman numerals

%----------------------------------------------------------------------------------------
%	OPENING PAGE
%----------------------------------------------------------------------------------------

%\makeatletter
%\extratitle{
%	% In the title page, the title is vspaced by 9.5\baselineskip
%	\vspace*{9\baselineskip}
%	\vspace*{\parskip}
%	\begin{center}
%		% In the title page, \huge is set after the komafont for title
%		\usekomafont{title}\huge\@title
%	\end{center}
%}
%\makeatother

%----------------------------------------------------------------------------------------
%	COPYRIGHT PAGE
%----------------------------------------------------------------------------------------

\makeatletter
\uppertitleback{\@titlehead} % Header

\lowertitleback{
	\textbf{Disclaimer}\\
	You can edit this page to suit your needs. For instance, here we have a no copyright statement, a colophon and some other information. This page is based on the corresponding page of Ken Arroyo Ohori's thesis, with minimal changes.
	
	\medskip
	
	\textbf{No copyright}\\
	\cczero\ This book is released into the public domain using the CC0 code. To the extent possible under law, I waive all copyright and related or neighbouring rights to this work.
	
	To view a copy of the CC0 code, visit: \\\url{http://creativecommons.org/publicdomain/zero/1.0/}
	
	\medskip
	
	\textbf{Colophon} \\
	This document was typeset with the help of \href{https://sourceforge.net/projects/koma-script/}{\KOMAScript} and \href{https://www.latex-project.org/}{\LaTeX} using the \href{https://github.com/fmarotta/kaobook/}{kaobook} class.
	
	The source code of this book is available at:\\\url{https://github.com/fmarotta/kaobook}
	
	(You are welcome to contribute!)
	
	\medskip
	
	\textbf{Publisher} \\
	First printed in May 2019 by \@publishers
}
\makeatother

%----------------------------------------------------------------------------------------
%	DEDICATION
%----------------------------------------------------------------------------------------

\dedication{
	The harmony of the world is made manifest in Form and Number, and the heart and soul and all the poetry of Natural Philosophy are embodied in the concept of mathematical beauty.\\
	\flushright -- D'Arcy Wentworth Thompson
}

%----------------------------------------------------------------------------------------
%	OUTPUT TITLE PAGE AND PREVIOUS
%----------------------------------------------------------------------------------------

% Note that \maketitle outputs the pages before here

% If twoside=false, \uppertitleback and \lowertitleback are not printed
% To overcome this issue, we set twoside=semi just before printing the title pages, and set it back to false just after the title pages
\KOMAoptions{twoside=semi}
\maketitle
\KOMAoptions{twoside=false}

%----------------------------------------------------------------------------------------
%	PREFACE
%----------------------------------------------------------------------------------------

%\chapter*{Preface}
\addcontentsline{toc}{chapter}{Preface} % Add the preface to the table of contents as a chapter

I am of the opinion that every \LaTeX\xspace geek, at least once during 
his life, feels the need to create his or her own class: this is what 
happened to me and here is the result, which, however, should be seen as 
a work still in progress. Actually, this class is not completely 
original, but it is a blend of all the best ideas that I have found in a 
number of guides, tutorials, blogs and tex.stackexchange.com posts. In 
particular, the main ideas come from two sources:

\begin{itemize}
	\item \href{https://3d.bk.tudelft.nl/ken/en/}{Ken Arroyo Ohori}'s 
	\href{https://3d.bk.tudelft.nl/ken/en/nl/ken/en/2016/04/17/a-1.5-column-layout-in-latex.html}{Doctoral 
	Thesis}, which served, with the author's permission, as a backbone 
	for the implementation of this class;
	\item The 
		\href{https://github.com/Tufte-LaTeX/tufte-latex}{Tufte-Latex 
			Class}, which was a model for the style.
\end{itemize}

The first chapter of this book is introductive and covers the most 
essential features of the class. Next, there is a bunch of chapters 
devoted to all the commands and environments that you may use in writing 
a book; in particular, it will be explained how to add notes, figures 
and tables, and references. The second part deals with the page layout 
and design, as well as additional features like coloured boxes and 
theorem environments.

I started writing this class as an experiment, and as such it should be 
regarded. Since it has always been indended for my personal use, it may 
not be perfect but I find it quite satisfactory for the use I want to 
make of it. I share this work in the hope that someone might find here 
the inspiration for writing his or her own class.

\begin{flushright}
	\textit{Federico Marotta}
\end{flushright}


%----------------------------------------------------------------------------------------
%	TABLE OF CONTENTS & LIST OF FIGURES/TABLES
%----------------------------------------------------------------------------------------

\begingroup % Local scope for the following commands

% Define the style for the TOC, LOF, and LOT
%\setstretch{1} % Uncomment to modify line spacing in the ToC
%\hypersetup{linkcolor=blue} % Uncomment to set the colour of links in the ToC
\setlength{\textheight}{23cm} % Manually adjust the height of the ToC pages

% Turn on compatibility mode for the etoc package
\etocstandarddisplaystyle % "toc display" as if etoc was not loaded
\etocstandardlines % toc lines as if etoc was not loaded

\tableofcontents % Output the table of contents

\listoffigures % Output the list of figures

% Comment both of the following lines to have the LOF and the LOT on different pages
\let\cleardoublepage\bigskip
\let\clearpage\bigskip

\listoftables % Output the list of tables

\endgroup

%----------------------------------------------------------------------------------------
%	MAIN BODY
%----------------------------------------------------------------------------------------

\mainmatter % Denotes the start of the main document content, resets page numbering and uses arabic numbers
\setchapterstyle{kao} % Choose the default chapter heading style

\pagelayout{wide} % No marginsCustomise this page according to your needs
\addpart{Eletricity}
\pagelayout{margin} % Restore margins

\setchapterpreamble[u]{\margintoc}
\chapter{Electrostatics}
\labch{intro}

\section{Electric Charge}

\section{Coulombs Law}

\section{The Electric Field}
	%TODO: Overview of the structure and components of the Nucleus.
	\subsection{Introduction}
		Electrodynamics main concern, as the name says, is the problem of given a known distribution of charge ${q_i}$ and a test charge $Q$. discover the value of the force exerted on the test charge.


	
	\subsection{Superposition Principle}
		In this section, we'll introduce one of Electromagnetism's main troubleshooter. \textit{The Superposition Principle}.\\
		First, one thing that needs to be clear is that this is a principle. It can be proven by the Maxwell's Equations (simply because they are linear). But history show that what happend was the other way around.\\%TODO: FIND REFERENCE FOR THIS.
		In physics, normally experimental facts motivates models (which then are tested experimentally for their predictions). In this case, Maxwell KNEW that his equations needed to be linear, in order for the superposition principle to work.\\
		The superposition principle states as follows:

	\subsection{Gauss Law}
		Lets think on a simple example. Suppose you have a pool covered with water where you can't see any sink or source, only the flow of water (don't ask me why). How can I know where are any of this two things located on the pool?\\
		Well, lets visually surround a given part of the pool (not disturbing the system) and then we'll watch for two things:\\
		\begin{itemize}
			\item If there is more water coming out of it than it's coming in, the water flux is clearly positive and then we have a source inside.
			\item Analogously, if there is more water coming in than coming out, the flux is negative and we have a sink.
		\end{itemize}
		But how do we quantify this idea? We know that it's proportional to something about the fluid (in this case, water) and something about the capacity of the source/sink. Lets return to the electrostatics example.
		We'll first define a concept called the electric flux as follows, Given a closed surface, we may define a normal vector to a element of area of this surface as $\text{d}\vec{S}$. Where is direction is perpendicular to the surface element (and with consensual direction pointing outwards) and with magnitude equal to the elements area. The flux is then given by:
		\begin{equation}
			\Phi(\vec{E}) = \oiint_S \vec{E} \cdot \text{d}\vec{S}
		\end{equation}
		At first, it seems like a rather abstract equation. Lets analyse it.\\
		First, lets see what the integrand has to tell us. It's composed of a scalar product between the electric field and a differential vector called $\text{d}\vec{S}$. So basically we're calculating what is the component of the electric field with respect to the normal vector of a element of area and summing up through the whole surface. So, what is the result of this calculation?\\
		Well, as we saw before, it must be directly proportional to the capacity of the charge inside (in this case, $q$) and something about the environments were studying. So, what can it be?\\
		Lets make for a simple charge located at the origin with a spherical surface.
		\begin{equation}
			\oint_S \vec{E} \cdot d\vec{A} = \int_S \frac{1}{4\pi\epsilon_0} \left(\frac{q}{r^2}\hat{r}\right)\cdot(r^2\sin\theta d\theta d\phi \hat{r}) = \frac{q}{\epsilon_0}
		\end{equation}
		Notice that the radius of the sphere vanishes, for while the surface area goes proportional to $r^2$, the field goes down with $1/r^2$.\\
		You may ask now. \textit{"How about other surfaces and other charge distributions?} Well, about the surfaces we can say that its flux is equal to the sphere and by the superposition principle we can extrapolate for any charge distribution. %TODO: prove this statement.


\section{Charge distribution}
	Charge distribution is an important concept on the theory of Electrostatics. It's normally expressed in three different ways:
	\begin{itemize}
		\item $1D \longrightarrow [\lambda] = C/m$
		\item $2D \longrightarrow [\sigma] = C/m^2$
		\item $3D \longrightarrow [\rho] = C/m^3$
	\end{itemize}
	The first two expressions are an approximation of the third one. (almost) every day-to-day object is three dimensional, however, if some dimensions are much larger than the remaining, we can approximate the object as some ($d<3$) d-dimensional object. Which is much easier to deal with.

	


\pagelayout{wide} % No marginsCustomise this page according to your needs
\addpart{Magnetism}
\pagelayout{margin} % Restore margins

\pagelayout{wide} % No margins
\addpart{Topics on Electromagnetism}
\pagelayout{margin} % Restore margins


\appendix % From here onwards, chapters are numbered with letters, as is the appendix convention

\pagelayout{wide} % No margins
\addpart{Appendix}
\pagelayout{margin} % Restore margins


%----------------------------------------------------------------------------------------

\backmatter % Denotes the end of the main document content
\setchapterstyle{plain} % Output plain chapters from this point onwards

%----------------------------------------------------------------------------------------
%	BIBLIOGRAPHY
%----------------------------------------------------------------------------------------

% The bibliography needs to be compiled with biber using your LaTeX editor, or on the command line with 'biber main' from the template directory

\defbibnote{bibnote}{Here are the references in citation order.\par\bigskip} % Prepend this text to the bibliography
\printbibliography[heading=bibintoc, title=Bibliography, prenote=bibnote] % Add the bibliography heading to the ToC, set the title of the bibliography and output the bibliography note

%----------------------------------------------------------------------------------------
%	NOMENCLATURE
%----------------------------------------------------------------------------------------

% The nomenclature needs to be compiled on the command line with 'makeindex main.nlo -s nomencl.ist -o main.nls' from the template directory

\nomenclature{$c$}{Speed of light in a vacuum inertial frame}
\nomenclature{$h$}{Planck constant}

\renewcommand{\nomname}{Notation} % Rename the default 'Nomenclature'
\renewcommand{\nompreamble}{The next list describes several symbols that will be later used within the body of the document.} % Prepend this text to the nomenclature

\printnomenclature % Output the nomenclature

%----------------------------------------------------------------------------------------
%	GREEK ALPHABET
% 	Originally from https://gitlab.com/jim.hefferon/linear-algebra
%----------------------------------------------------------------------------------------

\vspace{1cm}

{\usekomafont{chapter}Greek Letters with Pronounciation} \\[2ex]
\begin{center}
	\newcommand{\pronounced}[1]{\hspace*{.2em}\small\textit{#1}}
	\begin{tabular}{l l @{\hspace*{3em}} l l}
		\toprule
		Character & Name & Character & Name \\ 
		\midrule
		$\alpha$ & alpha \pronounced{AL-fuh} & $\nu$ & nu \pronounced{NEW} \\
		$\beta$ & beta \pronounced{BAY-tuh} & $\xi$, $\Xi$ & xi \pronounced{KSIGH} \\ 
		$\gamma$, $\Gamma$ & gamma \pronounced{GAM-muh} & o & omicron \pronounced{OM-uh-CRON} \\
		$\delta$, $\Delta$ & delta \pronounced{DEL-tuh} & $\pi$, $\Pi$ & pi \pronounced{PIE} \\
		$\epsilon$ & epsilon \pronounced{EP-suh-lon} & $\rho$ & rho \pronounced{ROW} \\
		$\zeta$ & zeta \pronounced{ZAY-tuh} & $\sigma$, $\Sigma$ & sigma \pronounced{SIG-muh} \\
		$\eta$ & eta \pronounced{AY-tuh} & $\tau$ & tau \pronounced{TOW (as in cow)} \\
		$\theta$, $\Theta$ & theta \pronounced{THAY-tuh} & $\upsilon$, $\Upsilon$ & upsilon \pronounced{OOP-suh-LON} \\
		$\iota$ & iota \pronounced{eye-OH-tuh} & $\phi$, $\Phi$ & phi \pronounced{FEE, or FI (as in hi)} \\
		$\kappa$ & kappa \pronounced{KAP-uh} & $\chi$ & chi \pronounced{KI (as in hi)} \\
		$\lambda$, $\Lambda$ & lambda \pronounced{LAM-duh} & $\psi$, $\Psi$ & psi \pronounced{SIGH, or PSIGH} \\
		$\mu$ & mu \pronounced{MEW} & $\omega$, $\Omega$ & omega \pronounced{oh-MAY-guh} \\
		\bottomrule
	\end{tabular} \\[1.5ex]
	Capitals shown are the ones that differ from Roman capitals.
\end{center}

%----------------------------------------------------------------------------------------
%	GLOSSARY
%----------------------------------------------------------------------------------------

% The glossary needs to be compiled on the command line with 'makeglossaries main' from the template directory

\newglossaryentry{computer}{
	name=computer,
	description={is a programmable machine that receives input, stores and manipulates data, and provides output in a useful format}
}

% Glossary entries (used in text with e.g. \acrfull{fpsLabel} or \acrshort{fpsLabel})
\newacronym[longplural={Frames per Second}]{fpsLabel}{FPS}{Frame per Second}
\newacronym[longplural={Tables of Contents}]{tocLabel}{TOC}{Table of Contents}

\setglossarystyle{listgroup} % Set the style of the glossary (see https://en.wikibooks.org/wiki/LaTeX/Glossary for a reference)
\printglossary[title=Special Terms, toctitle=List of Terms] % Output the glossary, 'title' is the chapter heading for the glossary, toctitle is the table of contents heading

%----------------------------------------------------------------------------------------
%	INDEX
%----------------------------------------------------------------------------------------

% The index needs to be compiled on the command line with 'makeindex main' from the template directory

\printindex % Output the index

%----------------------------------------------------------------------------------------
%	BACK COVER
%----------------------------------------------------------------------------------------

% If you have a PDF/image file that you want to use as a back cover, uncomment the following lines

%\clearpage
%\thispagestyle{empty}
%\null%
%\clearpage
%\includepdf{cover-back.pdf}

%----------------------------------------------------------------------------------------

\end{document}
